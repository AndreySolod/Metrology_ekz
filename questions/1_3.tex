\section{Измерения}

Прямое измерение – это измерение, измерение в котором искомое значение величины находят непосредственно из опытных данных.

Косвенное измерение – искомую величину находят по известной зависимости между искомой величиной и величинами, определяемыми прямыми измерениями.

Методы измерения:

Метод непосредственной оценки – значение величины определяется непосредственно по отсчетному устройству измерительного прибора.
Для этого необходимо, чтобы диапазон показаний шкалы был больше значения измеряемой величины.
\[  ДП>L  \]
При методе непосредственной оценки (НО) настройку прибора на нуль производят по базовой поверхности прибора. Под действием различных факторов (изменения температуры, влажности, вибраций и т.д.) может произойти смешение нуля. Поэтому периодически необходимо производить проверку и соответствующую регулировку.

Метод сравнения – измеряемую величину сравнивают с величиной, воспроизводимой мерой. При измерении методом сравнения с мерой результатом наблюдения является отклонение измеряемой величины от значения меры. Значение измеряемой величины получают алгебраическим суммированием значения меры и отклонения от этой меры, определенного по показанию прибора.

\[ L = M + П \]

Случайные погрешности измерения --- погрешность, изменяющая величину и знак в зависимости от случайных обстоятельств.

Случайные погрешности подчиняются закону распределения Гаусса:

\[ P(x) = \dfrac{1}{G \sqrt{2 \pi}} e^{-\frac{(x - \mu)^2}{2 G^2}} \]

Систематическая погрешность --- погрешность, постоянная по определённому закону при повторных применениях.

 Результаты наблюдений, полученные при наличии систематической погрешности, называются неисправленными. При проведении измерений стараются в максимальной степени исключить или учесть влияние систематических погрешностей. Это может быть достигнуто следующими путями:

\begin{itemize}
	\item устранением источников погрешностей до начала измерений. В большинстве областей измерений известны главные источники систематических погрешностей и разработаны методы, исключающие их возникновение или устраняющие их влияние на результат измерения. В связи с этим в практике измерений стараются устранить систематические погрешности не путем обработки экспериментальных данных, а применением СИ, реализующих соответствующие методы измерений;
	
	\item определением поправок и внесением их в результат измерения;
	
	\item оценкой границ неисключенных систематических погрешностей.
\end{itemize}

\section{Контроль}

Калибры – средства измерительного контроля, предназначенные для проверки соответствия действительных размеров, формы и расположения поверхностей деталей заданным требованиям.

Калибры применяют для контроля деталей в массовом и серийном производствах. Калибры бывают нормальные и предельные.

Нормальный калибр – однозначная мера, которая воспроизводит среднее значение (значение середины поля допуска) контролируемого размера. При использовании нормального калибра о годности детали судят, например, по зазорам между поверхностями детали и калибра, либо по «плотности» возникающего сопряжения между контролируемой деталью и нормальным калибром. Оценка зазора, следовательно, результаты контроля в значительной мере зависят от квалификации контролера и имеют субъективный характер.

Предельные калибры – мера или комплект мер обеспечивающие контроль геометрических параметров деталей по наибольшему и наименьшему предельным значениям. Изготавливают предельные калибры для проверки размеров гладких цилиндрических и конических поверхностей, глубины и высоты уступов, параметров резьбовых и шлицевых поверхностей деталей. Изготавливают также калибры для контроля расположения поверхностей деталей, нормированных позиционными допусками, допусками соосности и др.

При контроле предельными калибрами деталь считается годной, если проходной калибр под действием силы тяжести проходит, а непроходной калибр не проходит через контролируемый элемент детали. Результаты контроля практически не зависят от квалификации оператора.

По конструкции калибры делятся на пробки и скобы. Для контроля отверстий используют калибры-пробки, для контроля валов – калибры-скобы.

Обозначение на чертеже: $ ПР=60,0065_{-0.0012} $

\section{Техническое регулирование}

 Техническое регулирование – правовое регулирование отношений при установлении, применении и исполнении обязательных требований к объектам технического регулирования, добровольных требований к объектам технического регулирования, выполнению работ или оказанию услуг и правовому регулированию отношений при оценке соответствия.

Документ, осуществляющий это регулирование – технический регламент, документ, имеющий силу закона РФ, который и устанавливает обязательные для применения требования к объектам технич. регулирования.

Объекты технического регулирования:
\begin{itemize}
	\item продукция, в т.ч. здания, строения и сооружения;
	
	\item процессы проектирования (в т.ч. изыскания), производства, строительства, монтажа, наладки, эксплуатации, хранения, перевозки, реализации и утилизации.
\end{itemize}

 Техническое регулирование направлено на отношения, возникающие:

\begin{enumerate}
	\item  при разработке, принятии, применении и исполнении обязательных требований к объектам технического регулирования;
	
	\item  разработке, принятии, применении и исполнении на добровольной основе требований к объектам технического регулирования, выполнению работ или оказанию услуг;
	
	\item  оценке соответствия.
\end{enumerate}

Контроль (надзор) за соблюдением требований ТР – проверка выполнения юридическим лицом или индивидуальным предпринимателем требований ТР к продукции или связанным с ними процессам ЖЦП и принятие мер по результатам проверки.

\section{Стандартизация}

Принцип предпочтительности --- один из основных принципов, используемых в стандартизации. Различают качественный и количественный аспекты применения этого принципа. Качественный аспект состоит в образовании предпочтительных рядов объектов стандартизации. Предпочтительность устанавливают для сложных объектов (изделий, деталей, процессов, типовых решений, обозначений), а также для их элементов (отдельных требований, параметров, норм точности и т.д.). Количественный аспект связан с построением числовых параметрических рядов.

